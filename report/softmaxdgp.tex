%!TEX root = report.tex

\section{Deep models for clustering}

The second classical problem of unsupervised learning is clustering. \noteSR{Add something general on clustering}

\subsection{Extended \dgplvm}

The architecture already discussed in the previous sections can be also adapted to handle clustering assignment.

Our intent is to build a probabilistic model, such that for each observation $\yvect_i$ it gives a probability of being assigned to a cluster; formally we want to compute

\begin{equation}
    \tilde k = \arg\max_{k}p(x_{ij}=k\vert\yvect_i, \Xi)
\end{equation}

where $\Xi$ is the set of all the parameters and hyper-parameters of the model. This can be connected do a latent variable model, since the vector of label probability $\xvect_i$ is an hidden variable and it's never observed in the training set.

This is achieved keeping the same architecture presented in Figure \ref{fig:DGP:diagram} and replacing the first layer of \gp{s} with a \softmax layer. The \softmax layer simply implements a softmax function on each samples of the latent space $\Xvect$; in practice:

\begin{equation}
    \Pivect = \dfrac{\exp(\Xvect)} {\sum_{j=0}^{D\mathrm{lat} - 1} \exp(\Xvect_{:,j})}
\end{equation}

where $\Pivect$ is a $\nobs\times\Dlat$ matrix of probabilities and $\Dlat$ is the number of cluster imposed to the model.

\def\layersep{1.8cm}
\def\scale{0.5}
\definecolor{myblue}{rgb}{0.2, 0.2, 0.8}
\newcommand{\red}{\textcolor{red}}
\newcommand{\orange}{\textcolor{orange}}
\newcommand{\green}{\textcolor{green}}
\newcommand{\blue}{\textcolor{blue}}
\newcommand\colorpar[1]{\textcolor{myyellow}{#1}}
\newcommand\colordata[1]{\textcolor{myorange}{#1}}
\newcommand\colorlatent[1]{\textcolor{myblue}{#1}}
\newcommand\colorweights[1]{\textcolor{mygreen}{#1}}

\begin{figure}[t]
\begin{tikzpicture}[shorten >=1pt,->,draw=green!50!black, node distance=\layersep]
    \tikzstyle{every pin edge}=[<-,shorten <=1pt]
    \tikzstyle{neuron}=[circle,minimum size=10pt,inner sep=0pt, draw=black]

    \tikzstyle{data neuron}=[neuron, fill=yellow!50!red];
    \tikzstyle{output neuron}=[neuron, fill=blue!50];
    \tikzstyle{hidden neuron}=[neuron];
    \tikzstyle{annot} = [text width=4em, text centered];
    \tikzstyle{annot-weight} = [text centered, fill=white!20, rectangle, draw];

    %% % Draw the input layer nodes
    %% \foreach \name / \y in {1,...,4}
    %%     \node[data neuron, pin=left:{\scriptsize $X$ \#\y}] (F0-\name) at (0,-\y) {};

    % Draw the input layer nodes
    \foreach \name / \y in {1,...,4}
        \node[data neuron] (F0-\name) at (0,\scale*-\y) {};

    % Draw the first hidden layer nodes - Phi^{(0)}
    %\foreach \name / \y in {1,...,5}
    %    \path[yshift=\scale*0.5cm]
    %        node[hidden neuron] (Phi0-\name) at (\layersep,\scale*-\y cm) {};

    % Draw the layer node for F^{(1)}
    \foreach \name / \y in {1,...,4}
        \path[yshift=\scale]
            node[output neuron] (F1-\name) at (1 * \layersep,\scale*-\y) {};

    % Draw the second hidden layer nodes - Phi^{(1)}
    \foreach \name / \y in {1,...,5}
        \path[yshift=\scale*0.5cm]
            node[hidden neuron] (Phi1-\name) at (2 * \layersep,\scale*-\y cm) {};

    % Draw the layer node for F^{(1)}
    \foreach \name / \y in {1,...,3}
        \path[yshift=\scale*-0.5cm]
            node[output neuron] (F2-\name) at (3 * \layersep,\scale*-\y) {};

    \foreach \name / \y in {1,...,3}
        \path[yshift=\scale*-0.5cm]
            node[data neuron] (Y-\name) at (4 * \layersep,\scale*-\y) {};

    % Draw Thetas
    %\node[annot] (theta-0) at (0.5 * \layersep,\scale*-4.5) {\scriptsize\red{$\thetavect$}$^{(0)}$};
    %\node[annot] (theta-1) at (2.5 * \layersep,\scale*-4.5) {\scriptsize\red{$\thetavect$}$^{(1)}$};


    %% \node[output neuron,pin={[pin edge={->}]right:{\scriptsize $F^{(1)}$}}, right of=Phi1-3] (F2) {};


    % Connect every node in the input layer with every node in the
    % hidden layer.
    %\foreach \source in {1,...,4}
    %    \foreach \dest in {1,...,5}
    %        \path (F0-\source) edge (Phi0-\dest);

    % Connect every node in the hidden layer with the output layer
    \foreach \source in {1,...,4}
        \foreach \dest in {1,...,4}
            \path (F0-\source) edge (F1-\dest);

    % Connect every node in the hidden layer with the output layer
    \foreach \source in {1,...,4}
        \foreach \dest in {1,...,5}
            \path (F1-\source) edge (Phi1-\dest);

    % Connect every node in the hidden layer with the output layer
    \foreach \source in {1,...,5}
        \foreach \dest in {1,...,3}
            \path (Phi1-\source) edge (F2-\dest);

    % Connect every node in the hidden layer with the output layer
    \foreach \source in {1,...,3}
        \foreach \dest in {1,...,3}
            \path (F2-\source) edge (Y-\dest);

    % Annotate the layers
    \node[annot,above of=Phi0-1, node distance=0.6cm] (Layer-Phi0) {{\scriptsize $\colorlatent{\Pivect}$}};
    \node[annot,left of=Layer-Phi0] {{\scriptsize $\colordata{X}$}};
    %\node[annot,right of=Layer-Phi0] {{\scriptsize $\colorlatent{\Pivect}$}};
    \node[annot,above of=Phi1-1, node distance=0.6cm] (Layer-Phi1) {{\scriptsize $\Phi^{(1)}$}};
    \node[annot,right of=Layer-Phi1] (Layer-F2) {{\scriptsize $\colorlatent{F}^{(2)}$}};
    \node[annot,right of=Layer-F2] (Layer-Y) {{\scriptsize $\colordata{Y}$}};


    \node[annot-weight] (Omega-Phi0) at (0.5 * \layersep, \scale*-2.5) {{\scriptsize $\colorweights{\softmax}$}};
    %\node[annot-weight,right of=Omega-Phi0] (W-Phi0) {{\scriptsize $\colorweights{W}^{(0)}$}};
    \node[annot-weight] (Omega-Phi1) at (1.5 * \layersep, \scale*-2.5) {{\scriptsize $\colorweights{\Omega}^{(1)}$}};
    \node[annot-weight,right of=Omega-Phi1] (W-Phi1) {{\scriptsize $\colorweights{W}^{(1)}$}};

    %\path (theta-0) edge[red] (Omega-Phi0);
    %\path (theta-1) edge[red] (Omega-Phi1);

\end{tikzpicture}
\label{fig:DGP:diagram}
\end{figure}

Invariance to rotations and to initialisation

Probabilistic model
