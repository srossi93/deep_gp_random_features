%!TEX root = dgplvm.tex
\section{Conclusions}


\begin{frame}[allowframebreaks]{Things done}
  \metroset{block=fill}
  \begin{exampleblock}{What has been done so far?}
    \begin{itemize}
        \item Migration of existing code from \tensorflow 0.12 to \tensorflow 1.1
        \item Extension for \dgplvm for both dimensionality reduction and clustering in a new dedicated class
        \item Experiments on dimensionality reduction and clustering with both real and synthetic datasets
    \end{itemize}
  \end{exampleblock}

  \begin{alertblock}{Problems spotted}
    \begin{itemize}
        \item Due to \tensorflow computational graph's engine, it's impossible to optimize over \texttt{placeholders}, making the minibatch-based learning not straightforward
        \item The resulting model seems to be too much sensible to initialization pf hyperparameters (in particular, the \texttt{lengthscale})
    \end{itemize}
  \end{alertblock}
\end{frame}

\begin{frame}{Things to be done}
    \metroset{block=fill}
    \begin{alertblock}{Future works}
     	\begin{itemize}
     	    \item Scalability extension through the use of minibatch-based learning
            \item Further experiments on clustering with real dataset and some comparisons with state-of-the-art algorithms.
      	\end{itemize}
    \end{alertblock}
\end{frame}

\begin{frame}[standout]
  \huge Questions? 
\end{frame}


\begin{frame}[allowframebreaks] %allow to expand references to multiple frames (slides)

\frametitle{Bibliography}
\tiny
\scriptsize{\bibliographystyle{acm}}
\scriptsize
\bibliography{Bibliography} %bibtex file name without .bib extension
\nocite{*}
\end{frame}
