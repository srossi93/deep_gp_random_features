%!TEX root = dgplvm.tex

\section{Deep Gaussian Processes}
\label{cap:dgp}

\subsection{Gaussian Process review}
\begin{frame}{Gaussian Process - Weight space}
    A Gaussian Process can be seen as a Bayesian linear regression with possibly infinite basis functions.
    \begin{equation}
        \bar f(\xvectnew) = \phi(\xvectnew)^\top\wvect.
    \end{equation}
    \pause
    Introducing the covariance function $k(\xvect, \xvect')$, it can be proved that the equation above can be written as follows
    \begin{equation}
        \bar f(\xvectnew) = \kvect(\xvectnew)^\top \alphavect,
    \end{equation}
    where $\alphavect = K^{-1}\yvect$ and $\kvect(\xvectnew)$ denote the vector of covariances between the point $\xvectnew$ and the $\nobs$ training points.
\end{frame}

\begin{frame}{Fourier expansion}
    The popular RBF kernel can be approximated as follows
    \begin{equation}
        k_{\mathrm{rbf}}(\xvect_i, \xvect_j) \approx \frac{1}{N_{\mathrm{RF}}} \sum_{r=1}^{N_{\mathrm{RF}}} \zvect(\xvect_i | \tilde{\omegavect}_r)^{\top} \zvect(\xvect_j | \tilde{\omegavect}_r) \text{,}
    \end{equation}
    where $\zvect(\xvect | \omegavect) = [\cos(\xvect^{\top} \omegavect), \sin(\xvect^{\top} \omegavect)]^{\top}$ and with $\tilde{\omegavect}_{r} \sim p(\omegavect)$.
\end{frame}

\subsection{Deep Architecture}
\begin{frame}{Deep Architecture}
    \def\layersep{1.4cm}
\def\scale{0.8}
\definecolor{myblue}{rgb}{0.2, 0.2, 0.8}
\newcommand{\red}{\textcolor{red}}
\newcommand{\orange}{\textcolor{orange}}
\newcommand{\green}{\textcolor{green}}
\newcommand{\blue}{\textcolor{blue}}
\newcommand\colorpar[1]{\textcolor{myyellow}{#1}}
\newcommand\colordata[1]{\textcolor{myorange}{#1}}
\newcommand\colorlatent[1]{\textcolor{myblue}{#1}}
\newcommand\colorweights[1]{\textcolor{mygreen}{#1}}

\begin{figure}[t]
\begin{tikzpicture}[shorten >=1pt,->,draw=green!50!black, node distance=\layersep]
    \tikzstyle{every pin edge}=[<-,shorten <=1pt]
    \tikzstyle{neuron}=[circle,minimum size=10pt,inner sep=0pt, draw=black]

    \tikzstyle{data neuron}=[neuron, fill=yellow!50!red];
    \tikzstyle{output neuron}=[neuron, fill=blue!50];
    \tikzstyle{hidden neuron}=[neuron];
    \tikzstyle{annot} = [text width=4em, text centered];
    \tikzstyle{annot-weight} = [text centered, fill=white!20, rectangle, draw];

    %% % Draw the input layer nodes
    %% \foreach \name / \y in {1,...,4}
    %%     \node[data neuron, pin=left:{\scriptsize $X$ \#\y}] (F0-\name) at (0,-\y) {};

    % Draw the input layer nodes
    \foreach \name / \y in {1,...,4}
        \node[data neuron] (F0-\name) at (0,\scale*-\y) {};

    % Draw the first hidden layer nodes - Phi^{(0)}
    \foreach \name / \y in {1,...,5}
        \path[yshift=\scale*0.5cm]
            node[hidden neuron] (Phi0-\name) at (\layersep,\scale*-\y cm) {};

    % Draw the layer node for F^{(1)}
    \foreach \name / \y in {1,...,2}
        \path[yshift=\scale*-1.0cm]
            node[output neuron] (F1-\name) at (2 * \layersep,\scale*-\y) {};

    % Draw the second hidden layer nodes - Phi^{(1)}
    \foreach \name / \y in {1,...,5}
        \path[yshift=\scale*0.5cm]
            node[hidden neuron] (Phi1-\name) at (3 * \layersep,\scale*-\y cm) {};

    % Draw the layer node for F^{(1)}
    \foreach \name / \y in {1,...,3}
        \path[yshift=\scale*-0.5cm]
            node[output neuron] (F2-\name) at (4 * \layersep,\scale*-\y) {};

    \foreach \name / \y in {1,...,3}
        \path[yshift=\scale*-0.5cm]
            node[data neuron] (Y-\name) at (5 * \layersep,\scale*-\y) {};

    % Draw Thetas
    \node[annot] (theta-0) at (0.5 * \layersep,\scale*-4.5) {\scriptsize\red{$\thetavect$}$^{(0)}$};
    \node[annot] (theta-1) at (2.5 * \layersep,\scale*-4.5) {\scriptsize\red{$\thetavect$}$^{(1)}$};


    %% \node[output neuron,pin={[pin edge={->}]right:{\scriptsize $F^{(1)}$}}, right of=Phi1-3] (F2) {};


    % Connect every node in the input layer with every node in the
    % hidden layer.
    \foreach \source in {1,...,4}
        \foreach \dest in {1,...,5}
            \path (F0-\source) edge (Phi0-\dest);

    % Connect every node in the hidden layer with the output layer
    \foreach \source in {1,...,5}
        \foreach \dest in {1,...,2}
            \path (Phi0-\source) edge (F1-\dest);

    % Connect every node in the hidden layer with the output layer
    \foreach \source in {1,...,2}
        \foreach \dest in {1,...,5}
            \path (F1-\source) edge (Phi1-\dest);

    % Connect every node in the hidden layer with the output layer
    \foreach \source in {1,...,5}
        \foreach \dest in {1,...,3}
            \path (Phi1-\source) edge (F2-\dest);

    % Connect every node in the hidden layer with the output layer
    \foreach \source in {1,...,3}
        \foreach \dest in {1,...,3}
            \path (F2-\source) edge (Y-\dest);

    % Annotate the layers
    \node[annot,above of=Phi0-1, node distance=0.6cm] (Layer-Phi0) {{\scriptsize $\Phi^{(0)}$}};
    \node[annot,left of=Layer-Phi0] {{\scriptsize $\colordata{X}$}};
    \node[annot,right of=Layer-Phi0] {{\scriptsize $\colorlatent{F}^{(1)}$}};
    \node[annot,above of=Phi1-1, node distance=0.6cm] (Layer-Phi1) {{\scriptsize $\Phi^{(1)}$}};
    \node[annot,right of=Layer-Phi1] {{\scriptsize $\colorlatent{F}^{(2)}$}};
    \node[annot,above of=Y-1, node distance=1.35cm] (Layer-Y) {{\scriptsize $\colordata{Y}$}};


    \node[annot-weight] (Omega-Phi0) at (0.5 * \layersep, \scale*-2.5) {{\scriptsize $\colorweights{\Omega}^{(0)}$}};
    \node[annot-weight,right of=Omega-Phi0] (W-Phi0) {{\scriptsize $\colorweights{W}^{(0)}$}};
    \node[annot-weight] (Omega-Phi1) at (2.5 * \layersep, \scale*-2.5) {{\scriptsize $\colorweights{\Omega}^{(1)}$}};
    \node[annot-weight,right of=Omega-Phi1] (W-Phi1) {{\scriptsize $\colorweights{W}^{(1)}$}};

    \path (theta-0) edge[red] (Omega-Phi0);
    \path (theta-1) edge[red] (Omega-Phi1);

\end{tikzpicture}
\caption{The \dgp approximation used for building the deep model. At each hidden layer \gp{s} are replaced by their two-layer weight-space approximation. Random-features $\Phi^{(l)}$ are obtained using a weight matrix $\Omega^{(l)}$. This is followed by a linear transformation parameterized by weights $W^{(l)}$. } 
\label{fig:DGP:diagram}
\end{figure}

    This is the approximation of \dgp where
    \begin{equation}
        \Phi_{\mathrm{rbf}}^{(l)} = \sqrt{\frac{(\sigma^2)^{(l)}}{N_{\mathrm{RF}}^{(l)}}} \left[ \cos\left(F^{(l)} \Omega^{(l)}\right), \sin\left(F^{(l)} \Omega^{(l)}\right) \right] \text{,}
    \end{equation}
    and
    \begin{equation}
        F^{(l+1)} = \Phi_{\mathrm{rbf}}^{(l)} W^{(l)} \text{.}
    \end{equation}
\end{frame}
